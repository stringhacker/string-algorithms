\documentclass[a4paper,12pt]{article}
\usepackage{amsmath}
\usepackage{amsthm}
\usepackage{amssymb}
\usepackage{algorithm}
\usepackage{algorithmic}
\usepackage{polski}
\usepackage[utf8]{inputenc}

\DeclareMathOperator{\SA}{SA}
\DeclareMathOperator{\LCP}{LCP}

\title{A simple algorithm for Lempel-Ziv factorization}
\author{*}

\begin{document}

\maketitle

Faktoryzacja Lempel-Ziv'a dla słowa $w$ jest takim rozkładem $u_0 u_1 ... u_k = w$,
że każde $u_i$, za wyjątkiem możliwie ostatniego,
jest albo najdłużsym prefiksem $u_i u_{i + 1} ... u_k$ i występuje jako podsłowo w $u_0 u_1 ... u_i$,
ale nie tylko jako sufiks,
albo jest pojedynczym symbolem, gdy takiego prefiksu nie ma.

Authorzy proponują algorytm pozwalający obliczać faktoryzację w czasie liniowym i pamięci $o(n)$.
Jeszcze poprzedni wynik tych samych autorów osiągał liniowy czas i pamięć,
natomiast różnica pomiędzy dużym $O(n)$ tamtego algorytmu, i małym $o(n)$ dzisiejszego, jest na tyle istotna,
że nowy algorytm został opublikowany.

Algorytm ten, tak jak i poprzedni, korzysta z tablicy Longest Previous Factor.
Aby zrozumieć co to jest, weźmy taki najdłużsy czynnik słowa $w[1..i]$, równy $m$.
Wtedy $m$ musi być najdłużsym podsłowem słowa $w[1..i + |m| - 1]$,
i to jego długość będzie występować w tej tablicy, na pozycji $i$-tej.

Gdy posiadamy tablicę LPF, wyznaczanie faktoryzacji nie jest trudne.
Łatwo zauważyć, że ``najdłuższy poprzedni czynnik'', to dokładnie taki czynnik jakiego potrzebujemy do faktoryzacji.
Wystarczy zatem przejść po tablicy LPF zwracając kolejne czynniki,
pomijając przy tym czynniki pośrednie, występujące pomiędzy tymi z faktoryzacji,
oraz zamieniając wszystkie zera na jedynki w tablicy LPF, ponieważ faktoryzacja nie zawiera słów pustych.
\textbf{Algorithm 1} jest implementacją powyższej logiki.

\begin{algorithm}
\caption{lempel\_ziv\_factorization}
\begin{algorithmic} 
\REQUIRE LPF, n
\ENSURE LZ
\STATE LZ $\gets [\;]$
\STATE pos $\gets$ 1
\WHILE{pos $\leq$ n}
\STATE push(max(1, LPF[pos]), LZ)
\STATE pos $\gets$ pos + max(1, LPF[pos])
\ENDWHILE
\end{algorithmic}
\end{algorithm}

Pozostaje wyznaczenie LPF. Do tego korzystamy z tablic SA, i LCP --
z uporządkowanej tablicy sufiksów i tablicy najdłuższych prefiksów między nimi.
Nie będziemy projektować algorytmów do policzenia tych dwóch tablic,
gdyż wiele takich istnieje, i skupimy się na tym, jak z SA i LCP dostać LPF.

\pagebreak

Do policzenia tablicy SA proponowano jest użyć którykolwiek z wcześniej znalezionych liniowych algorytmów, w tym z
\begin{enumerate}
\item ``Simple linear work suffix array construction'' autorów J. Kärkkäinen i P. Sanders,
\item ``Linear-time longest-common-prefix computation in suffix arrays and its applications'' autorów T. Kasai, G. Lee, H. Arimura i S. Arikawa,
\item i ``Constructing suffix arrays in linear time'' autorów D.K. Kim, J.S. Sim, H. Park i K.Park,
\end{enumerate}
pamięciowa złożoność których jest usprawniona w pracy ``Space efficient linear time construction of suffix arrays'' autorstwa P. Ko i S. Aluru.

W przypadku LCP, algorytm z pracy ``Two space-saving tricks for linear-time LCP computation'' autora G. Manzini,
też jest wystarczająco dobry, aby nie wpłynąć na złożoność algorytmu LPF.



\end{document}
