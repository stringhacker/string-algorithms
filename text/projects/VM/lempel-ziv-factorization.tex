\documentclass[a4paper,12pt]{article}
\usepackage{amsmath}
\usepackage{amsthm}
\usepackage{amssymb}
\usepackage{algorithm}
\usepackage{algorithmic}
\usepackage{polski}
\usepackage[utf8]{inputenc}

\DeclareMathOperator{\SA}{SA}
\DeclareMathOperator{\LCP}{LCP}

\title{A simple algorithm for Lempel-Ziv factorization}
\author{*}

\begin{document}

\maketitle

Faktoryzacja Lempel-Ziv'a dla słowa $w$ jest takim rozkładem $u_0 u_1 ... u_k = w$,
że każde $u_i$, za wyjątkiem możliwie ostatniego,
jest albo najdłużsym prefiksem $u_i u_{i + 1} ... u_k$ i występuje jako podsłowo w $u_0 u_1 ... u_i$,
ale nie tylko jako sufiks,
albo jest pojedynczym symbolem, gdy takiego prefiksu nie ma.

Algorytm korzysta z tablicy Longest Previous Factor.
Aby zrozumieć co to jest, weźmy najdłużsy czynnik słowa $w[1..i]$, równy $m$.
Wtedy $m$ musi być najdłużsym podsłowem słowa $w[1..i + |m| - 1]$,
a jego długość będzie na pozycji $i$-tej tej tablicy.

Gdy posiadamy tablicę LPF, to wyznaczanie faktoryzacji nie jest trudne,
ponieważ ``najdłuższy poprzedni czynnik'', to dokładnie taki czynnik jakiego potrzebujemy do faktoryzacji.
Czyli, wystarczy przejść po tablicy LPF zwracając kolejne czynniki,
pomijając przy tym czynniki pośrednie, występujące pomiędzy tymi z faktoryzacji.
Algorytm zatem wygląda następująco:

\begin{algorithmic} 
\REQUIRE LPF
\ENSURE LZ
\STATE $pos \gets 1$
\STATE $s \gets \Phi[i]$
\WHILE{$text[i+l] = text[s+l]$}
\STATE $l \gets l + 1$
\ENDWHILE
\STATE $\PLCP[i] \gets l$
\STATE $l \gets \max(l - 1, 0)$
\end{algorithmic}

\end{document}
